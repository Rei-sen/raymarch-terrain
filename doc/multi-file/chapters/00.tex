\subsubsection*{Tytuł pracy} 
\Title Renderowanie terenu metodą raymarching.
\hyphenpenalty=10000
\emergencystretch=\maxdimen

\subsubsection*{Streszczenie}  
%% (Streszczenie pracy – odpowiednie pole w systemie APD powinno zawierać kopię tego streszczenia.)
Praca przedstawia sposób proceduralnej generacji terenu oraz technikę renderowania raymarching.
Zaprezentowano działanie metody raymarching, porównując ją do innych metod, jej wady oraz zalety. Wskazano różne techniki generacji terenu, w~szczególności opisano rozwiązanie oparte o ułamkowe ruchy Brown'a, które zostało wykorzystane w~programie. Ponadto pokazano modyfikacje algorytmu raymarching pozwalające na~renderowanie cieni czy chmur. Zarówno generacja terenu jak i renderowanie zostały zaimplementowane jako jednostka cieniująca karty graficznej.

%% Praca przedstawia program pozwalający na tworzenie proceduralnie generowanego terenu. Uzyskany teren renderowanu jest z wykorzystaniem metody raymarching.
%% Program pozwalający na tworzenie proceduralnie generowanego terenu oraz renderujący go przy pomocy metody raymarching.
\subsubsection*{Słowa kluczowe} 
Grafika komputerowa, ray marching, renderowanie

\subsubsection*{Thesis title} 
\begin{otherlanguage}{british}
\TitleAlt Rendering terrain using raymarching method.
\end{otherlanguage}

\subsubsection*{Abstract} 
\begin{otherlanguage}{british}
%% (Thesis abstract – to be copied into an appropriate field during an electronic submission – in English.)
This work presents a method of procedural terrain generation as well as a rendering technique called raymarching.
It presents an explanation of raymarching, comparing it to other methods, its advantages and disadvantages. Various terrain generation techniques are described, in particular, a solution based on fractional Brownian motion, which was used in the program. In addition, modifications of the raymarching algorithm that allow rendering shadows or clouds are shown. Both terrain generation and rendering were implemented as an OpenGL shader.
%% Program capable of creating procedurally generated terrain and rendering it using raymarching method.
\end{otherlanguage}
\subsubsection*{Key words}  
\begin{otherlanguage}{british}
Computer graphics, ray marching, rendering
\end{otherlanguage}

