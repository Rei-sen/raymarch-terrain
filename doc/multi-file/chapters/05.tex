\chapter{Specyfikacja wewnętrzna}
\label{ch:05}

\section{Architektura systemu oraz przedstawienie idei}
Do realizacji projektu wykorzystano język C++ oraz bibliotekę OpenGL. Do kompilacji programu wykorzystano system budowy CMake. Kod odpowiadający za renderowanie obrazu napisany został w języku GLSL i wykonywany jest na karcie graficznej.
W pierwszej kolejności następuje utworzenie okna programu z wykorzystaniem biblioteki GLFW. Kolejny krok to inicjalizacja biblioteki OpenGL przy wykorzystaniu biblioteki GLAD i rozpoczęcie głównej pętli programu, w której obsługiwane są następujące czynności:
\begin{itemize}
\item obsługa klawiatury i myszki,
\item wyświetlenie i obsługa interfejsu użytkownika,
\item pobranie ustawień z interfejsu użytkownika i przekazanie ich do karty graficznej,
\item rozpoczęcie renderowania obrazu,
\item wyświetlenie uzyskanego obrazu.
\end{itemize}

Informacje o konfiguracji przekazywane są do jednostek cieniujących przy wykorzystaniu zmiennych typu \ang{uniform}. Jednostka cieniująca wierzchołków tworzy prostokąt pokrywający cały ekran, na którym odbywa się renderowanie obrazu. Całość renderowania odbywa się w jednostce cieniującej fragmentów. Pierwszym etapem jest obliczenie początkowej pozycji wiązki światła oraz jej kierunek dla każdego piksela. Uzyskane w ten sposób informacje o wiązce światła przekazywane są do fukncji \ang{render}. Funkcja ta odpowiada za następujące czynności:
\begin{itemize}
\item wywołanie funkcji algorytmu raymarching,
\item uzyskanie wektorów normalnych,
\item obliczenie koloru w zależności od rodzaju przeciętej wiązką światłą powierzchni,
\item obliczenia związane z cieniem i cieniowaniem,
\item obliczenie końcowego koloru danego piksela,
\item zastosowanie korekcji gamma.
\end{itemize}

\section{Przegląd ważniejszych klas oraz struktur danych}
W programie jedną najważniejszych jest klasa \ang{Application}, która odpowiada za inicjalizację całego programu oraz wykorzystanych bibliotek. Klasa ta zawiera główną pętlę programu, dodatkowo odpowiada za obsługę klawiatury i myszki. Deklaracja tej klasy została przedstawiona na rysunku \ref{fig:c++:application}.

\begin{figure}[H]
\centering
\begin{lstlisting}[language=C++]
class Application {
public:
  Application();
  ~Application();

  void run();
  void processInput();

private:
  Window win;
  UI ui;
  Config config;
  Renderer renderer;
};
\end{lstlisting}
\caption{Deklaracja klasy \ang{Application}.}
\label{fig:c++:application}
\end{figure}

Równie ważną jest klasa \ang{Renderer}, która odpowiada za inicjalizację podstawowych elementów niezbędnych do rozpoczęcia renderowania, kompilację jednostek cieniujących oraz ich połączenie. Ponadto umożliwia przekazywanie konfiguracji do jednostek cieniujących i rozpoczęcie renderowania. Rysunek \ref{fig:c++:renderer} przedstawia deklarację tej klasy.

\begin{figure}[H]
\centering
\begin{lstlisting}[language=C++]

class Renderer {
public:
  Renderer();
  ~Renderer();

  void update();
  void draw(const Config &c);

private:
  void updateUniforms(const Config &c);

  std::unique_ptr<GL::Program> program;
  std::unique_ptr<GL::Shader> vertexShader;
  GL::VAO vao;
};

\end{lstlisting}
\caption{Deklaracja klasy \ang{Renderer}.}
\label{fig:c++:renderer}
\end{figure}

Należy przytoczyć jeszcze klasę \ang{Window}, odpowiadającą za inicjalizację oraz obsługę okna aplikacji.
Deklaracja tej klasy widoczna jest na rysunku \ref{fig:c++:window}.

\begin{figure}[H]
\centering
\begin{lstlisting}[language=C++]
class Window {
public:
  Window();
  Window(unsigned width, unsigned height);
  ~Window();

  bool shouldClose() const;
  void pollEvents();
  void swapBuffers();

  operator GLFWwindow *() const { return w; }

private:
  GLFWwindow *w;
};
\end{lstlisting}
\caption{Deklaracja klasy \ang{Renderer}.}
\label{fig:c++:window}
\end{figure}

\section{Przegląd ważniejszych algorytmów}

Algorytm \ang{raymarching} jest najważniejszym z algorytmów wykorzystanych w programie. Jego rola polega na stopniowym
przesuwaniu wiązki światła do momentu przecięcia z dowolnym punktem powierzchni sceny. Sam algorytm wykorzystuje podane parametry, którymi są początkowa pozycja wiązki światła oraz jej kierunek. Jego końcowy wynik to odległość, którą musi przebyć wiązka by preciąć najbliższą powierzchnię. Algorytm ten wykorzystuje pomocniczą funkcję
\ang{map()}, która zwraca odległość między podanym punktem a najbliższą powierzchnią sceny. Kod tej funkcji zmienia się w zależności od renderowanej sceny, dlatego pominięto jej pseudokod.
Rysunek \ref{fig:pseudokod:raymarching} przedstawia pseudokod tego algorytmu.

\begin{figure}[H]
\centering
\begin{lstlisting}[language=C]
float raymarch(vec3 rayOrigin, vec3 rayDirection) {
  float distance = 0.0;

  for (int i = 0; i < MAX_STEPS; i++) {
    vec3 pos = rayOrigin + rayDirection * distance;
    float d = map(pos);
    distance += d;

    if (abs(d) < EPSILON) {
      return distance;
    }
    if (distance > MAX_DIST) {
      break;
    }
  }

  return MAX_DIST;
}
\end{lstlisting}
\caption{Pseudokod algorytmu renderowania metodą \ang{raymarching}.}
\label{fig:pseudokod:raymarching}
\end{figure}

Aby uzyskać efekt cienia, wykorzystano modyfikację tego algorytmu, w punkcie przecięcia wiązki wysyłana nowa wiązka skierowana w kierunku źródła światła. Na podstawie informacji o przecięciu ze sceną tej wiązki można określić czy dany punkt jest oświetlony. Dodatkowo wykorzystując wzór \ref{eq:r-shadow} można uzyskać efekt ,,miękkich'' cieni przy niewielkim koszcie obliczeń \cite{bib:iqsoftshadow}.
\begin{equation}
\label{eq:r-shadow}
r = min(\frac{map(p)}{\text{distance}})
\end{equation}
Rysunek \ref{fig:pseudokod:shadow} przedstawia pseudokod z wykorzystaniem ,,miękkich'' cieni.
\begin{figure}[H]
\centering
\begin{lstlisting}[language=C]

float shadow(vec3 rayOrigin) {

  float distance = 0.0;
  float minR = MAX_DIST;

  for (int i = 0; i < MAX_STEPS; i++) {
    vec3 pos = ro + sunDir * distance;
    float d = map(pos);

    float r = d/depth;
    minR = min(r, minR);

    depth += d;

    if (abs(d) < EPSILON)
      return 0.0;

    if (d > MAX_DIST) break;
  }
  return smoothstep(0.0, 1.0, minR);
}
\end{lstlisting}[H]
\caption{Pseudokod modyfikacji algorytmu \ang{raymarching} wykorzystany do obliczenia cieni.}
\label{fig:pseudokod:shadow}
\end{figure}

Różnice wynikające z zastosowania wzoru \ref{eq:r-shadow} przedstawia rysunek \ref{fig:shadow-comp}

\begin{figure}[H]
\centering
\subfloat[zwykłe cienie]{
  \includegraphics[width=0.5\textwidth]{./graf/shadow.png}
}
\subfloat[,,miękkie'' cienie]{
  \includegraphics[width=0.5\textwidth]{./graf/shadowsmooth.png}
}
%% \includegraphics[width=0.65\textwidth]{./graf/plot/bilinear.png}
\caption{Porównanie wyników metod renderowania cieni}
\label{fig:shadow-comp}
\end{figure}


%% \subchapter{Szczególy implementacji wybranych fragmentów kodu}
%% \begin{i%% temize}
%% \item przedstawienie idei
%% \item architektura systemu
%% \item komponenty, moduły, biblioteki, przegląd ważniejszych klas (jeśli występują)
%% \item opis struktur danych (i organizacji baz danych)
%% \item przegląd ważniejszych algorytmów (jeśli występują)
%% \item szczegóły implementacji wybranych fragmentów, zastosowane wzorce projektowe
%% \item diagramy UML
%% \end{itemize}

% % % % % % % % % % % % % % % % % % % % % % % % % % % % % % % % % % %
% Pakiet minted wymaga odkomentowania w pliku config/settings.tex   %
% importu pakietu minted: \usepackage{minted}                       %
% i specjalnego kompilowania:                                       %
% pdflatex -shell-escape praca                                      %
% % % % % % % % % % % % % % % % % % % % % % % % % % % % % % % % % % %


%% Krótka wstawka kodu w linii tekstu jest możliwa, np.  \lstinline|int a;| (biblioteka \texttt{listings})% lub  \mintinline{C++}|int a;| (biblioteka \texttt{minted})
%% .
%% Dłuższe fragmenty lepiej jest umieszczać jako rysunek, np. kod na rys \ref{fig:pseudokod:listings}% i rys. \ref{fig:pseudokod:minted}
%% , a naprawdę długie fragmenty – w załączniku.


%% \begin{figure}
%% \centering
%% \begin{lstlisting}
%% class test : public basic
%% {
%%     public:
%%       test (int a);
%%       friend std::ostream operator<<(std::ostream & s,
%%                                      const test & t);
%%     protected:
%%       int _a;

%% };
%% \end{lstlisting}
%% \caption{Pseudokod w \texttt{listings}.}
%% \label{fig:pseudokod:listings}
%% \end{figure}

%% %\begin{figure}
%% %\centering
%% %\begin{minted}[linenos,frame=lines]{c++}
%% %class test : public basic
%% %{
%% %    public:
%% %      test (int a);
%% %      friend std::ostream operator<<(std::ostream & s,
%% %                                     const test & t);
%% %    protected:
%% %      int _a;
%% %
%% %};
%% %\end{minted}
%% %\caption{Pseudokod w \texttt{minted}.}
%% %\label{fig:pseudokod:minted}
%% %\end{figure}
