\chapter{Wstęp}
\label{ch:wstep}
\hyphenpenalty=10000
\emergencystretch=\maxdimen
%% \begin{itemize}
%% \item wprowadzenie w~problem/zagadnienie
%% \item osadzenie problemu w~dziedzinie
%% \item Wprowadzenie
%% \section{Wprowadzenie}
Przedmiotem niniejszej pracy jest przedstawienie sposobu renderowania metodą ray marching dla~problemu renderowania trójwymiarowego terenu.
%% jest techniką modelowania światła, pozwalająca na~generowanie obrazów...
%% technika oparta
Ray marching jest~to~technika generowania obrazów dla~dwuwymiarowych oraz trójwymiarowych scen\cite{bib:gcraymarch}.
Działanie metody ray marching, podobnie do metody ray tracing, polega na~odnalezieniu przecięcia wiązek światła wraz~z~renderowaną sceną.
Rozwiązania oparte o~wiązki światła dość dobrze odwzorowują rzeczywistość, dzięki czemu można w~dość prosty sposób uzyskać fotorealistyczny obraz.
Techniki tego typu świetnie radzą sobie z~problemami takimi jak odbicia czy~załamania światła, których efekt znacznie trudniej jest osiągnąć w~rozwiązaniach opartych o~rasteryzacje trójkątów.
Największą wadą tego rodzaju renderowania jest ich złożoność obliczeniowa. W~najprostszym przypadku z~kamery wysyłana jest jedna wiązka odpowiadająca każdemu pikselowi, dla~której trzeba obliczyć przecięcia z całą sceną.
Czas działania znacząco pogarsza się wraz~z~implementacją cieniowania, odbić światła, przezroczystości, załamania światła, efektów wolumetrycznych i innych działań poprawiających jakość oraz realizm tworzonego obrazu.
Uwzględniając ilość wykonywanych obliczeń, metody renderowania oparte o~wiązki światła nie nadają się do renderowania obrazów w~czasie rzeczywistym, przy użyciu dostępnego obecnie sprzętu.
Rozwiązania tego typu idealnie nadają się jednak do zastosowań, w~których nie ma takich ograniczeń czasowych jak np. obrazy czy~filmy.

Technika ray tracing jest analitycznym rozwiązaniem problemu znalezienia przecięcia wiązki z~powierzchnią.
Oznacza to, że punkt przecięcia można uzyskać przez rozwiązanie wzoru matematycznego, opisującego to przecięcie.
Pomimo istnienia analitycznych rozwiązań przecięć z~niektórymi powierzchniami (np. z~kulą, z~trójkątem), brak jest rozwiązania
pozwalającego na~proste znalezienie przecięcia z~dowolną powierzchnią.

Alternatywą dla~metod analitycznych jest technika ray marching. Jest~to~numeryczna metoda estymacji przecięć
wiązki światła z~dowolną powierzchnią\cite{bib:gpugems2}\cite{bib:spheretracing}.
Technika ta polega na~sukcesywnym przesuwaniu wiązki światła o~daną odległość do momentu przecięcia z~powierzchnią.
Rozwiązanie to jest dość wolne oraz~niedokładne, jednak istnieją różnego rodzaju optymalizacje
pozwalające poprawić dokładność oraz szybkość algorytmu\cite{bib:Liktor2008RayTI} (takie~jak: sphere tracing, bounding spheres, itp.).

%% \item cel pracy
%% \section{Cel pracy}
Celem pracy jest utworzenie aplikacji renderującej proceduralnie generowany teren. Użytkownik posiada możliwość poruszania kamerą oraz zmiany parametrów wpływających na~generacje oraz renderowanie terenu.

W przedmiotowym zakresie pracy zostanie przedstawiona metoda proceduralnego generowania
terenów, która może być wykorzystywana w~obszarach związanych z~tworzeniem filmów, gier komputerowych, animacji itp. Metoda ta zostanie zaimplementowana na~karcie graficznej, w języku programowania GLSL (\ang{OpenGL Shading Language}) jako jednostka cieniująca fragmentów.

W dalszej części pracy zostaną ujęte wymagania funkcjonalne, niefunkcjonalne oraz~wskazane zostaną narzędzia użyte podczas tworzenia programu.
Opisane zostaną metody instalacji programu na różnych systemach operacyjnych, jego sposób obsługi oraz~zostanie przedstawiony przykład działania programu dla osiągnięcia planowanych efektów.

W pracy zostaną przedstawione graficzne wyniki zastosowania różnych algorytmów oraz ich parametrów. W~dalszej części zostanie poddany analizie sposób działania wykorzystanych algorytmów.


%% \item zakres pracy
%% \section{Zakres pracy}
%% \item zwięzła charakterystyka rozdziałów
%% \section {Charakterystyka rozdziałów}
%% \item jednoznaczne określenie wkładu autora, w~przypadku prac wieloosobowych – tabela z~autorstwem poszczególnych elementów pracy
%% \end{itemize}
